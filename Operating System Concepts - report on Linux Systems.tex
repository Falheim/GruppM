
\documentclass[12pt]{article} 

\usepackage[utf8]{inputenc}
\usepackage[english]{babel}

\usepackage{geometry} 
\geometry{a4paper}

\usepackage{fancyhdr} 
\pagestyle{fancy} 
\renewcommand{\headrulewidth}{0pt}
\lhead{}\chead{}\rhead{}
\lfoot{}\cfoot{\thepage}\rfoot{}

\usepackage{sectsty}
\allsectionsfont{\sffamily\mdseries\upshape}

%============================================================================================================
\begin{document}
\title{Operating Systems - Report on the Linux Systems}
\author{Samuel, Andersson, Johan Dahlberg, Eric Falheim, \\Camilla Heiding, Xuan Hoang, Amer Hodzic}
\date{Today}
\maketitle

\newpage
\tableofcontents
\newpage

\section{Linux History and Licensing} %To be written by


\section{Kernel}
The kernel is the core of the operating system and make communication possible to the hardware.

There can exist many kernels in the same system. And in the case of a failure, when updating a kernel,
you can always boot the system with an older version that was working.

\subsection{Kernel Modules} %To be written by
%man-pages fork()
A kernel module is a driver that can be loaded into the kernel dynamically, at boot-time or run-time,
to make communication between a device and hardware possible (via the kernel).
This makes the kernel more lightweight due to the fact that unused modules can be unloaded from the kernel, which frees up space.

All modules can be found in the /lib/modules/ folder.
Working with modules is easy in Linux. There are commands you can run from the CLI to load, unload and list modules.
\begin{itemize}
  \item lsmod - lists the loaded modules.
  \item modprobe \textbf{[-r]} \textit{module\_name} - will load or unload named module depending on the flag -r.
  \item insmod \textit{module\_name} - same as using modprobe without -r flag.
  \item rmmod \textit{module\_name} - same as using modprobe with -r flag.
\end{itemize}


\section{Process Management} %To be written 
A process is associated with a Process ID (PID) on creation, which will be unique for this process. 
Every process is spawned by a parent process with a fork, 
and on creation the child-process memory space is identical to the parent's, but separate.

A process can run in the foreground or in the background.

\section{Scheduling} %To be written by

\section{Memory Management} %To be written by

\section{File Systems} %To be written by

\subsection{Mounting}
There are no disk-drives in Linux, instead all devices are mounted on a mounting points in the filesystem. 
So installing a new hard-disk or inserting a USB-drive is treated much the same way,
 by mounting them on a node where you later want to access it.

\subsection{Structure}
  The filesystem begins at the root /, where folders that divide the system logically is placed. Some examples:
  \begin{itemize}
    \item /bin - Contains binaries for the system. Easily accessed by the system via environment variable \$PATH.
    \item /boot - Contains files for booting the system, including the kernel.
    \item /dev - Contains the system's devices.
    \item /etc - Contains configuration files for the system.
    \item /lib - Modules, software libraries and information databases.
    \item /home - Home directory for every user on the system. 
    \item /proc - 
    \item /tmp - A temporary directory that is emptied periodically, or when the system is shut down.
  \end{itemize}
 
\section{Input and Output} %To be written by

\section{Interprocess Communication} %To be written by

\section{Network Communications} %To be written by

\section{Security} %To be written by

\end{document}

